\documentclass{article}


\usepackage[preprint]{neurips_2020}

\usepackage[utf8]{inputenc} % allow utf-8 input
\usepackage[T1]{fontenc}    % use 8-bit T1 fonts
\usepackage{hyperref}       % hyperlinks
\usepackage{url}            % simple URL typesetting
\usepackage{booktabs}       % professional-quality tables
\usepackage{amsfonts}       % blackboard math symbols
\usepackage{nicefrac}       % compact symbols for 1/2, etc.
\usepackage{microtype}      % microtypography

\title{Report Template for APML25/26@UU}

% TODO: Include the assignment number, group name, group members
\author{Name 1, Name 2, Name 3, Name 4 \\Group 0}

\begin{document}

\maketitle

\begin{abstract}
% Write a brief summary (150-250 words) of the study’s purpose, methodology, main findings, and implications. Highlights the problem, approach, key results, and conclusions.
\end{abstract}

\section{Introduction}
% \begin{itemize}
% \item Background: Briefly introduce the assignment and establish its importance within the context of the field.
% \item Problem Statement: Define the data science problem in your own words.
% \item Objectives: Clearly state the purpose and objectives of your tasks.
% \item Structure of the Paper: A sentence or two summarizing the paper’s structure helps readers navigate it.
% \end{itemize}

\section{Data}
% \begin{itemize}
% 	\item Provide important descriptions about the data set(s) used. For example, what is the data set about? How many instances? How many features? What is the data type of each feature? Any other significant characteristics? 
% 	\item Use the data visualizations you created to discuss the characteristics of the data that are relevant for your tasks. 
% \end{itemize}
% 

\section{Methods and Experimental Setups}

% \begin{itemize}
% 	\item Experimental Setup / Model Selection: Describe the framework, models, or algorithms chosen or designed, and justify their selection. For example, which algorithms have you selected? Please also provide the reasons for your method selection. 
% 	\begin{itemize}
% 		\item (When preprocessing is relevant) Also describe the preprocessing steps and any transformations applied. For example, which data preprocessing steps are conducted? What was your feature engineering or feature selection process? 
% 	\end{itemize}
% 	\item Implementation Details: Outline technical details, parameters, libraries, or software packages used. Include sufficient information for reproducibility.
% 	\item Evaluation Metrics: Define the metrics for evaluating model performance and justifying their relevance to the assignment.
% \end{itemize}

This is an example of citation~\cite{DBLP:journals/ker/KurganM06}.



\section{Results and Discussion}
% \begin{itemize}
% 	\item Model Performance: Present model results, often using tables, graphs, or figures for clarity.
% 	\item Interpretation of Results: Provide insights into what the results mean in the context of the data science questions.
% 	\item Implications: Explain how the findings contribute to the task, addressing both theoretical and practical implications.
% \end{itemize}

\section{Conclusions}
% \begin{itemize}
% 	\item Summarize the key findings and contributions of the study.
% 	\item Suggest potential future directions, improvements, or alternative approaches.
% \end{itemize}

%\section*{References}
\bibliographystyle{plain}
\bibliography{references}


\end{document}